\documentclass{article}
\usepackage{mathtools}
\usepackage{amsmath}
\usepackage{amssymb}
\usepackage{amsfonts}
\usepackage{graphicx}
\usepackage{float}
\usepackage{multirow}
\usepackage{verbatim}

\linespread{1.3}
\setlength{\parindent}{0em}
\setlength{\parskip}{1em}
\setcounter{secnumdepth}{0}
\setcounter{MaxMatrixCols}{20}
\renewcommand{\arraystretch}{1.5}

\newcommand{\ts}{\textsuperscript}
\newcommand{\diff}{\mathop{}\!\mathrm{d}}
\newcommand{\prob}{\mathbb{P}}
\newcommand{\expect}{\mathbb{E}}
\newcommand{\var}{\text{Var}}

\DeclarePairedDelimiter{\abs}{\lvert}{\rvert}
\DeclarePairedDelimiter\norm{\lVert}{\rVert}
\DeclarePairedDelimiter\p{\lparan}{\rparan}

\title{Lilypond}
\author{Joshua Hwang}

\begin{document}
\section{Brief overview, stream of consciousness}
One of the results of this summer research has been the
development and extension of the classic Poisson Lilypond model.

We will not describe the Lilypond model here and it is assumed the
reader has a brief understanding of it.

The extensions we intended to introduce were the non-homogeneous
placement which was achieved using thinning.
A negative of this method is that we need a maximum which we initially
give an upper bound with.

Another extension was the changing of rates. We had previously had
each grain growing at equal rates. We now produce a random rate
(currently) based off independent uniform distributions.

Finally we implement Ebert and Last (2015) extension of random birth
times. The grains develop at different times in the cycle based
(currently) off independent uniform distributions. If the grain
begins growing \emph{after} another grain has grown on top it does not
get realised.

From these extensions we define the growing radii to be,
\begin{align*}
    r_i
    &=
    \begin{cases}
        v_i \times (t - t_{i0}) \quad &[t_{i0}, \infty) \\
        0 &\text{otherwise} \\
    \end{cases}
\end{align*}

With $D$ being the distance between the grains,
the question of collision time becomes,
\begin{align*}
    0 &= |D - r_j - r_k| \qquad [\max\{t_{j0}, t_{k0}\}, \infty) \\
    0 &= D - r_j - r_k \\
    &= D - v_j \times (t - t_{j0}) - v_k \times (t - t_{k0}) \\
    D &= v_j \times (t - t_{j0}) + v_k \times (t - t_{k0}) \\
    D &= v_j \times t - v_j \times t_{j0} + v_k \times t - v_k \times t_{k0} \\
    D + v_j t_{j0} + v_k t_{k0} &= (v_j + v_k)t \\
    t &= \frac{D + v_j t_{j0} + v_k t_{k0}}{v_j + v_k} \\
\end{align*}

Where $t$ is the global time.

This is only for the case where two grains are growing. In the case where the
one of the grains ($k$) is stationary we have,
\begin{align*}
    0 &= |D - r_j - r_k| \qquad [t_{j0}, \infty) \\
    0 &= D - v_j \times (t - t_{j0}) - r_k \\
    0 &= D - v_j t + v_j t_{j0} - r_k \\
    t &= \frac{D + v_j t_{j0} - r_k}{v_j} \\
\end{align*}

One issue that came up while working on this is determining if a collision will
ever occur. Keep in mind that our model allows a growing grain to smother an
unborn one.

A growing grain ($j$) will smother an unborn one ($k$) if the time of
smothering $t_{\text{smother}}$ occurs before the time $k$ gets born
then we know that grain will die.
\begin{align*}
    D &= v_j (t_{\text{smother}} - t_j) \\
    \frac{D}{v_j} &= t_{\text{smother}} - t_j \\
    t_{\text{smother}} &= \frac{D}{v_j} + t_j \\
    \text{thus} \\
    t_{\text{smother}} < t_k
\end{align*}

I first implemented the model in Python, \texttt{sim.py},
this was because speed was initially
not considered a factor. The program 

We're currently also using numeric integrals \texttt{dblquad} to calculate the 
integrals. This has the issues of numeric errors.

\section{Percolation}
A question that constantly arises with these models is the idea of percolation.

I claim the extensions we have made to the model do not alter the fact that the
model will percolate.

Here is a hand-wavy proof: We note our model rests in a continuous space thus
no two collisions will happen at the same time. This has converted our problem
into a graph problem involving cycles. If a cycle exists then percolation isn't
possible.

We recognise that two growing grains
can generate a collision and turn both stationary,
a growing grain and a stationary grain can
also generate a collision but two stationary grains cannot.
We consider a collision of two grains to be equivalent to creating a line
between the centres of each grain.

\subsection{ZZZ}
I'm struggling to go further with this type of proof.

\end{document}
