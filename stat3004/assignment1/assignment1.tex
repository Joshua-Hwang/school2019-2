\documentclass{article}
\usepackage{mathtools}
\usepackage{amsmath}
\usepackage{amssymb}
\usepackage{amsfonts}
\usepackage{graphicx}
\usepackage{float}
\usepackage{multirow}
\usepackage{verbatim}

\linespread{1.3}
\setlength{\parindent}{0em}
\setlength{\parskip}{1em}
\setcounter{secnumdepth}{0}
\setcounter{MaxMatrixCols}{20}
\renewcommand{\arraystretch}{1.5}

\newcommand{\ts}{\textsuperscript}
\newcommand{\diff}{\mathop{}\!\mathrm{d}}
\newcommand{\prob}{\mathbb{P}}
\newcommand{\expect}{\mathbb{E}}
\newcommand{\var}{\text{Var}}

\DeclarePairedDelimiter{\abs}{\lvert}{\rvert}
\DeclarePairedDelimiter\norm{\lVert}{\rVert}
\DeclarePairedDelimiter\p{\lparan}{\rparan}

\title{STAT3004 Assignment 1}
\author{Joshua Hwang (44302650)}

\begin{document}
\maketitle
\section{Q1}
We will attempt to show that $\var(X) = \expect \var(X|Y) + \var(\expect[X|Y])$

We first note some useful facts and tools:
\begin{align*}
    \var(X) &= \expect(X^2) - {(\expect X)}^2 \\
    \expect(aX+B) &= a\expect(X) + \expect(B) \\
    \expect(\expect(X|Y)) &= \expect(X) \\
    \expect{(X|Y)}^2 &= \expect(X^2|Y)
\end{align*}

These results have been proven in class and we shall take them as given
for this question.

\begin{align*}
    \expect \var(X|Y) + \var(\expect[X|Y])
    &= \expect\left(\expect({[X|Y]}^2) - {(\expect [X|Y])}^2\right)
        + \expect\left({\left(\expect[X|Y]\right)}^2\right) - {(\expect (\expect[X|Y]))}^2 \\
    &= \expect\left(\expect\left({[X|Y]}^2\right)\right) - \expect\left({(\expect [X|Y])}^2\right)
        + \expect\left({(\expect[X|Y])}^2\right) - {(\expect (\expect[X|Y]))}^2 \\
    &= \expect\left(\expect\left({[X|Y]}^2\right)\right) - {(\expect (\expect[X|Y]))}^2 \\
    &= \expect\left(\expect\left({[X^2|Y]}\right)\right) - {(\expect (X))}^2 \\
    &= \expect\left(X^2\right) - {(\expect (X))}^2 \\
    &= \var(X) \\
\end{align*}

\section{Q2}
\subsection{a}
We should remember that,
\begin{align*}
    \prob(X \geq x) &= \int_x^\infty f(t) \diff t \\
\end{align*}

With this (and Fubini's theorem) we can solve this,
\begin{align*}
    \int_0^\infty \prob(X \geq x) \diff x
    &= \int_0^\infty \int_x^\infty f(t) \diff t \diff x \\
    &= \int_0^\infty f(t) \int_0^t 1 \diff x \diff t
    & \text{Swap integrals} \\
    &= \int_0^\infty t f(t) \diff t \\
    &= \expect X & \text{Definition of expectation}
\end{align*}

\subsection{b}
We now extend our result for $\expect\left[X^\alpha\right]$.
\begin{align*}
    \int_0^\infty \alpha x^{\alpha-1} \prob(X \geq x) \diff x
    &= \int_0^\infty \int_x^\infty \alpha x^{\alpha-1} f(t) \diff t \diff x \\
    &= \alpha \int_0^\infty f(t) \diff t \int_0^t x^{\alpha-1} \diff x \diff t
    & \text{Swap integrals} \\
    &= \int_0^\infty t^\alpha f(t) \diff t \\
    &= \expect \left[X^\alpha\right] & \text{Definition of expectation}
\end{align*}

\section{Q3 NOT DONE YET}
% we show the first case G_1(z) is actually G(z)
% this is done by showing something???
% then you prove the second statement by G(G(G(G(z)))) which can be
% compressed into what you want
% expand using the law of total expectation
% sub S_i somehow into S_{i-1}
We note from lectures that $S_1 = G(z)$ and $S_0 = 1$. This first step
obviously satisfies the requirement since,
\begin{align*}
    G_0(z) &= z \\
    G_1(z) &= G(G_0(z)) = G_0(G(z)) \\
\end{align*}

We take a more general case. Note that since we're looking at a branching
process, $S_{k+1} = \sum_{i=1}^{S_k} X_{i,k+1}$. Where $X_{i,k+1}$ are iid.
\begin{align*}
    G_k(z) &= \expect z^{S_k} \\
    &= \expect \expect\left[ \prod_{i=1}^{S_{k-1}} z^{X_{i,k}} \middle| S_{k-1} \right] \\
    &= \expect \left[ \prod_{i=1}^{S_{k-1}} z^{X_{i,k}} \right] \\
    &= \expect \left[ \prod_{i=1}^{S_{k-1}} G(z) \right] \\
    &= \expect \left[ G{(z)}^{S_{k-1}} \right] \\
    &= G_{k-1}\left( G(z) \right) \\
\end{align*}

From here you can notice that the pattern is recursive. We note that our
initial case, $S_1 \to G(z)$. Each iteration afterwards composes the
previous PGF with $G(z)$. Thus we expand our formula into,
\begin{align*}
    G_{k-1}(z) &= G(\ldots G(z) \ldots ) & \text{Done $k-1$ times} \\
    G_k(z) &= G(\ldots G(z) \ldots ) & \text{Done $k$ times} \\
    &= G(G_{k-1}(z)) \\
\end{align*}

Thus both versions $G(z) = G(G_{k-1}(z)) = G_{k-1}(G(z))$ have been proven.

\section{Q4}
\subsection{a}
The offspring distribution is either 2 or 0 with probability. Thus we can
model this as Bernoulli.

\begin{align*}
    X &\sim 2 \times Ber(p)
\end{align*}

The PGF is given by,
\begin{align*}
    G(z) &= \expect z^X \\
    G(z) &= \expect \expect \left[z^X \middle| X \right] \\
    &= \expect \left[z^2 \times p + z^0 \times (1-p)\right] \\
    &= \expect \left[z^2 p + 1 - p\right] \\
    &= z^2 p - p + 1 \\
\end{align*}

\subsection{b}
From the lectures we know $\expect[S_n] = \mu^n$ and
$\var(S_n) = \mu^{n-1} \sigma^2 + \mu^2 \var(S_{n-1})$.

Where $\mu = \expect X$ and $\sigma^2 = \var X$.

As a piecewise function variance is
\begin{align*}
    \var(S_n) &= \sigma^2 \mu^{n-1} \frac{1-\mu^n}{1-\mu}
    & \text{When $\mu \neq 1$} \\
    \var(S_n) &= \sigma^2 n
    & \text{When $\mu = 1$}
\end{align*}

We find how $\mu$ changes with $p$
\begin{align*}
    \expect X = \mu &= 2 \times p + 0 \times (1 - p) \\
    &= 2 \times p \\
\end{align*}

It's clear that $p = 0.5$ it is critical, $p > 0.5$ is subcritical and
$p < 0.5$ is supercritical.

\subsection{c}
From the lectures we know we're looking for a fixed point,
$\eta = G(\eta)$. Where $\eta$ is
the chance of ultimate extinction and $G(z)$ is the PGF of the offspring
distribution.
\begin{align*}
    \eta &= G(\eta) \\
    \eta &= \eta^2 p + 1 - p \\
    0 &= \eta^2 p - \eta + 1 - p \\
    0 &= (\eta - 1)(p \eta + p - 1)  \\
\end{align*}

Our final solution now has $\eta = 1$ and $\eta = \frac{1-p}{p}$.
Certain chance is not an interesting result (and in fact it is an unstable
equilibrium) but our $\eta = \frac{1-p}{p}$ gives a stable probability
for the chance of ultimate extinction.

\section{Q5}
\subsection{a}
PGF is $G(z) = \expect \left[z^X\right]$.
\begin{align*}
    \expect \left[z^X\right]
    \expect \expect \left[z^X \middle| X \right] \\
    &= \expect \left[pz^A + (1-p)z^B\right] \\
    &= p \expect\left[z^A\right] + (1-p) \expect \left[z^B\right] \\
    &= p G_A + (1-p) G_B \\
\end{align*}

\section{Q6}
\subsection{a}
To create a branching process we must have an offspring distribution that
remains constant each generation. In the $B_n$ case this is not possible
since the popping of balloons in the previous generation has a limiting
effect on the balloons that can be popped presently (i.e.\ the origin
balloon can no longer be popped thus the first generation can at most pop
3 balloons).

We simplify (at the same time giving it a strict upper bound) by considering
the case where multiple ballloons are present at each location. Each generation
a single ballooon on that point will pop. It will also be possible for multiple
balloons to pop the same location. This ensure that offspring distribution
remains constant. Since each balloon can pop any of its neighbours with
chance $p$ independent of each other we model our offspring
$X \sim \text{Bin}(p,4)$. The rest of our model is standard for a branching
process model.

This new model will also be a strict upper bound for the scenario. This is
because the number of balloons that can be popped in the $S_n$ model is always
4. But in our $B_n$ model the number of balloons that can be popped may be
less than that, much like in the $B_1$ generation where each offspring
can only pop up to 3 balloons.

\subsection{b}
We must determine at what probability our model is guaranteed to stop popping.
Based on our knowledge of the branching process model we know this takes place
when $\mu < 1$, where $\mu$ is the expectation of the offspring distrubution
$X$.

\begin{align*}
    \expect X = 4\times p &= \mu \\
    4\times p < 1  \\
    p < \frac{1}{4}  \\
\end{align*}

The popping is guaranteed to stop when $p < \frac{1}{4}$.

\end{document}
