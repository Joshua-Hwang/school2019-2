\documentclass{article}
\usepackage{mathtools}
\usepackage{amsmath}
\usepackage{amssymb}
\usepackage{amsfonts}
\usepackage{graphicx}
\usepackage{float}
\usepackage{multirow}
\usepackage{verbatim}

\linespread{1.3}
\setlength{\parindent}{0em}
\setlength{\parskip}{1em}
\setcounter{secnumdepth}{0}
\setcounter{MaxMatrixCols}{20}
\renewcommand{\arraystretch}{1.5}

\newcommand{\ts}{\textsuperscript}
\newcommand{\diff}{\mathop{}\!\mathrm{d}}
\newcommand{\prob}{\mathbb{P}}
\newcommand{\expect}{\mathbb{E}}
\newcommand{\var}{\text{Var}}

\DeclarePairedDelimiter{\abs}{\lvert}{\rvert}
\DeclarePairedDelimiter\norm{\lVert}{\rVert}
\DeclarePairedDelimiter\p{\lparan}{\rparan}

\title{Exam prep}
\author{Joshua Hwang (44302650)}

\begin{document}
Each generation has a distribution
\begin{align*}
    X &\sim Bin(2,0.5)
\end{align*}

The PGF of our $X$ is,
\begin{align*}
    G(s) &= \sum \prob(k) s^k \\
    &= 1/4 \times s^0 + 1/2 \times s^1
    + 1/4 \times s^2 \\
    &= 1/4 + s/2 + s^2/4 \\
\end{align*}

Each generation is defined by,
\begin{align*}
    \eta_0 &= 0 \\
    \eta_1 &= G(0) \\
    &= 1/4 \\
    \eta_2 &= G(1/4) \\
    &= 1/4 + (1/4)/2 + (1/4)^2/4 \\
    &= 1/4 + 1/8 + 1/64 \\
    &= 16/64 + 8/64 + 1/64 \\
    &= 25/64 \\
\end{align*}

We extend this for the infinite case.
\begin{align*}
    G(\eta) &= \eta \\
    1/4 + \eta/2 + \eta^2/4 &= \eta \\
    1/4 - \eta/2 + \eta^2/4 &= 0 \\
    1 - 2\eta + \eta^2 &= 0 \\
    (\eta - 1)^2 &= 0 \\
\end{align*}

\end{document}
