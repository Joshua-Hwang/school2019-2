\documentclass{article}
\usepackage[a4paper, total={7in, 10in}]{geometry}
\usepackage{mathtools}
\usepackage{amsmath}
\usepackage{amssymb}
\usepackage{amsfonts}
\usepackage{graphicx}
\usepackage{float}
\usepackage{multirow}
\usepackage{multicol}
\usepackage{verbatim}
\usepackage{hyperref}

\linespread{1.3}
\setlength{\parindent}{0em}
\setlength{\parskip}{1em}
\setcounter{secnumdepth}{0}
\setcounter{MaxMatrixCols}{20}
\renewcommand{\arraystretch}{1.5}

\newcommand{\ts}{\textsuperscript}
\newcommand{\diff}{\mathop{}\!\mathrm{d}}
\newcommand{\prob}{\mathbb{P}}
\newcommand{\expect}{\mathbb{E}}
\newcommand{\var}{\text{Var}}

\DeclarePairedDelimiter{\abs}{\lvert}{\rvert}
\DeclarePairedDelimiter\norm{\lVert}{\rVert}
\DeclarePairedDelimiter\p{\lparan}{\rparan}

\title{Auric Enterprises Threat and Vulnerability Analysis}
\author{Joshua Hwang (44302650)}

\begin{document}
\maketitle
\begin{abstract}
    As operational technologies (OTs) and more specifically
    industry control systems (ICSs) continue to become more connected these
    industries begin to face the same attacks and threats
    that plague the business
    world.
    Because of this, older SCADA devices
    and other equipment used specifically in OTs have not been designed with
    cyber security in mind.
    This report aims to analyse the Auric Enterprises operations.
    We explore three of the most relevant threats to both Auric and OTs in
    general; malware, remote code execution and data theft.
    We find
    legacy devices and poor network security as well as many other
    vulnerabilities within the Auric work site could manifest into
    realised threats. We then provide important controls and
    recommendations that aim to mitigate the risk of these threats being
    realised.
\end{abstract}

\begin{multicols}{2}
    \section{Introduction}
    This report aims to analyse the Auric Enterprises operations.
    The three threats that will be analysed will be malware, 
    remote code execution and data theft. This report will provide a brief
    description of each threat followed by the vulnerabilities in Auric's operations
    that can be exploited to allow a breach. We then provide potential
    controls that aim to mitigate the risk of these threats being
    realised.

    \section{Malware}
    Malware remains one of the most common forms of attacks with 41.2\% of ICS
    computers detecting malware \cite{kapersky}.
    Malware (short for malicious software) is a piece of software designed to attack
    the system it resides on. 
    Globally,
    Kapersky determined up to 25\% of all malware was conventional (non ICS) trojan
    horses.
    For ICS computers specifically they determined that the most common form of
    malware was worms via a removable device.
    They also found that the main sources for these attacks were from
    the internet (roughly 25\%) and removable devices (i.e. USB) with roughly 8\%
    \cite{kapersky}.

    Malware, once inside a machine,
    can spread itself throughout a system. The malware can then
    do a variety of things. 
    In the case of worms they may spread themselves to every
    computer on the network and then deploy additional malware to the computers
    they have infected.
    
    One such example is the Stuxnet worm which aimed to damage Iran's nuclear
    program. Stuxnet was found in 2010 but it had likely been hidden for much
    longer. Stuxnet exploited several zero-day 
    exploits in both the Windows operating
    system, Siemens Step7 SCADA devices and S7 PLCs \cite{stuxnet}.

    \subsection{Vulnerabilities}
    In Auric's network configuration, it is very possible for malware to spread
    throughout the system once inside.
    Since the network is flat it only takes one compromised device to gain access to
    all devices on the system \cite{forbes}.

    Additionally, 
    Windows XP has not been supported by Microsoft since 2014 \cite{windowsxp}.
    Thus Microsoft will no longer provide security updates for this version of their
    operating system. 

    \subsection{Realised threat}
    One potential attack is
    the Stuxnet worm which was mentioned previously. The devices in the Auric
    operations site are exactly the same ones exploited by Stuxnet; Siemens
    Step7 SCADA devices. Though such exploits have been patched in the latest
    and final
    version of Windows XP it's possible that exploits discovered after 2014 
    could be used to replace the Windows part of Stuxnet.

    Consider this other potential attack.
    An interested attacker may make use of a zero day exploit.
    Since it is a zero day exploit no security patch has been released to
    counter the exploit. that gains some
    kind of entry into the system.
    Due to the flat network,
    it's now very easy for the malware to travel
    through the network. Deploying ransomware on the finance data or even disrupting
    the SCADA systems.

    \subsection{Controls}
    We recommend updating the operating system of the
    engineering workstations to ensure
    the Stuxnet worm and other possible exploits have a lower chance of gaining
    access into our system.

    Segmenting the network will also protect separate sections of the system
    from being compromised thus reducing the impact of successful attacks.

    In the case that Auric's databases get compromised via ransomware.
    It's recommended that Auric stores backups of their
    data secure and external to any
    network removing any chance of an attack via a network.

    %\section{Human error or sabotage}
    %IBM found that 48\% of all breaches in 2018 were
    %caused by malicious insiders and 27\% of breaches were caused by human error
    %\cite{ibm}. CA technologies affirms this with 53\% of companies confirming they
    %had experienced an insider attack in the past year \cite{ca}.

    %An insider can either use their privileges in the system or find ways to
    %escalate
    %their own to either leak information or disrupt the current systems.
    %Though malicious insiders have motives,
    %an accidental insider could be just as
    %damaging.

    %\subsection{Vulnerabilities}
    %A major problem is the CEO has given their login and
    %password to their personal assistant. Not only does this give the personal
    %assistant complete access and control over the operation system,
    %but attributes
    %all actions of this account to the CEO.
    %If an external attacker was to obtain access to the CEO's account the
    %personal assistant could become unfairly suspected.
    %Further suspected due to the conflict of interest the PA has between
    %their national home and the Auric.

    %The personal assistant makes all the arrangements for the CEO.
    %The lack of English may hinder communication and increase the possibility of
    %human error.

    %The CEO has unlimited access to all the systems on site. Considering the CEO
    %"travel[s] extensively for business" it's possible that they are
    %ignorant of the
    %nuanced situations happening on the ground floor.
    %It would be much more
    %preferable for the engineers and other workers on site to make the decisions and
    %monitor the situation. [A less fierce tone (ask Jack)]
    %This also breaches the need-to-know principle.

    %\subsection{Realised threat}
    %Consider the following scenario.
    %The very busy CEO makes a passing remark to the personal assistant. Due to the
    %language barrier the PA mishears the instruction and begins releasing
    %information accidentally. Since the CEO's account has full access to the
    %operations systems no one on the ground floor is able to prevent the actions
    %from passing through. The vital information is lost and Auric may lose the
    %\$10,000,000 for that year from their largest customer. 
    %After this disastrous event, any attempt to investigate the incident will point
    %straight to the PA and their association with Kamaria.

    %\subsection{Controls}
    %Remove as many privileges as possible from the CEO (and other members) and
    %establish a more secure way of enacting decisions.
    %Perhaps via phone call or person to person.
    %If the personal assistant is required to do work for the CEO the PA should get
    %their own account ensuring a level of accountability on their part.

    \section{Remote code execution}
    Remote code execution (or RCE)
    is the threat that an external entity gains access to a
    system and is able to run arbitrary code or commands on a machine inside the
    system.

    In an industrial control system context this is extremely dangerous due to
    amount of
    powerful machinery and vital safety equipment present on site. 
    If such equipment
    was to become compromised the health and safety of workers on site could
    become endangered. For example the Stuxnet worm mentioned previously
    could have caused nuclear
    meltdown if desired \cite{stuxnet}.

    In April 2019 an attack against a water utility was conducted via access through
    the Remote Desktop protocol available on Windows machines \cite{kapersky}.
    The attackers aim was to deploy ransomware on their data but many other avenues
    of attack could have been realised.

    \subsection{Vulnerabilities}
    Telnet allows users to run a terminal and execute commands when away from the
    machine. In the wrong hands this ability can authorise malicious individuals the
    same abilities.
    Telnet was a protocol designed in 1969 but by the 1990s 
    it was no longer equipped to
    handle the new requirements of the modern net.
    Authentication and encryption were not
    part of many telnet implementations \cite{telnet}. Not only, 
    but it is in the
    top ten target ports due to the number of exploits found in telnet and its
    implementations \cite{telnet}.

    A study was done on SCADA devices and found "more than 60 categories of errors
    were discovered, the majority of which were in the form of incorrect error
    responses to malformed traffic." \cite{shaky}. These protocol vulnerabilities
    can allow the attacker to induce segmentation
    faults to buffer, heap and stack overflows \cite{shaky} \cite{scada}.

    There are certain machines on-site that use unencrypted (open) Wi-Fi. The
    justification of their use was
    that the machines remain far from the boundaries of the
    work site. But since there has been no mention of patrols on the work site
    there is no guarantee that a dedicated attacker would not consider trespassing
    and hiding in nearby vegetation. To gain access to these unencrypted
    machines.

    \subsection{Realised threat}
    Consider the following possible scenario,
    an attacker may trespass and enter through the native vegetation at which point
    they could begin communicating with the open Wi-Fi devices. From this point
    they have gained access to and control of a piece of heavy machinery
    that could be used to
    disrupt operations or even cause major damage to the facilities.

    Additionally
    in public settings, say an employee attempts to telnet into the site in the
    comfort of an airplane lounge. She decides to use the public Wi-Fi for
    convenience. Since telnet is unencrypted malicious entities on the network
    may sniff the packets and discover confidential information and
    even how to access
    the network. From there the malicious user
    now has a way to access on-site machinery and execute commands freely.

    \subsection{Controls}
    Instead of using the dated 
    and vulnerable telnet protocol we strongly suggest
    that Auric makes the shift to the Secure SHell (SSH) protocol. SSH, and its
    implementation, are freely available and even built-in to Windows 10.
    SSH offers many secure features that Telnet lacked; authentication,
    encryption and even RSA keys \cite{ssh}.

    As stated in the report, refitting is expensive and requires expensive
    downtime. In this case we appeal to the more practical solution of hiring
    guards to patrol the border of the site. Though the machinery is still
    vulnerable we can mitigate the chance of attack by making physical access
    less viable.

    \section{Data theft}
    Since proprietary data is so important to Auric business,
    losing the exclusivity of this
    data would be detrimental.
    Thales reports that roughly 50\% of companies have
    experienced a data breach and 17\% to 34\% have experienced one in the past year
    \cite{thales}.

    Data theft can occur if databases are accessed or if networks are packet
    sniffed.
    Access to the database has already been discussed in the previous sections
    so we will mainly focus on packet sniffing.

    \subsection{Vulnerabilities}
    Wired Equivalent Privacy (or WEP)
    is no longer considered secure. A paper published in 2001 showcased a
    debilitating weakness in the RC4 stream cipher; the cipher used in WEP
    \cite{rc4}.

    Using this technique among other exploits in WEP implementations FBI agents were
    able to use publicly available equipment and software to 
    gain access into a WEP network in under three minutes \cite{fbi}. Since this
    occurred in 2005 it is safe to say that WEP is no longer secure.

    As mentioned in the malware section of this analysis it is possible for an
    attacker to get close enough to the network to begin attempts at accessing it.
    This becomes much worse when we recall that some of the devices are using
    unencrypted Wi-Fi.
    Moreover, Wi-Fi repeaters are used throughout the site and naively echo
    all traffic across the site making it even easier for attackers to
    eavesdrop.

    Even without using the WEP exploit that the FBI used. 
    The password used is not secure. The type of password generated by the famous
    xkcd comic does not hold the test of time. A common technique in
    password cracking is
    dictionary attacks. Such attacks assume that the password is a set of real
    words and uses this to narrow down the password search space \cite{xkcd}.

    \subsection{Realised threat}
    Consider the following scenario,
    an attacker stands close to the boundary of the work site. Though this 
    person has not broken into the property,
    the WEP network is reachable from where they're standing.
    
    Making use of the either the WEP FBI cracking method or a dictionary attack
    against the password they gain access to the network in only a few minutes.
    Once inside the network the attacker can begins packet sniffing.
    The attacker 

    Since the Wi-Fi repeaters echo all Wi-Fi traffic they receive,
    from his vantage point outside the operations area the attacker
    can freely listen to all activity in the work site.

    \subsection{Controls}
    Once again we recommend establishing patrols around the border of the site
    to deter attackers.

    We also recommend promoting the security protocol of as many network
    connected devices
    as possible.

    Finally, we advise that the Wi-Fi repeaters be replaced with
    or modified so they can become devices that
    allow for network segmentation. Thus even if an attacker successfully gains
    access to the network close to the border they will not be able to sniff
    all traffic on site, merely a small subset of it.

    \section{Conclusions}
    This paper has explored three of the most relevant threats to both Auric
    and OTs in general; malware, remote code execution and
    data theft. Through our analysis of Auric we found
    legacy devices and poor network security as well as many other
    vulnerabilities within the Auric work site could manifest into
    realised threats.
    Our suggestion is the upgrade as many legacy devices and legacy software as
    possible. We also recommend creating data backups and
    segmenting the network to ensure successful
    attacks have lower impact.

    \bibliographystyle{IEEEtran}
    \bibliography{IEEEabrv, report.bib}
\end{multicols}
\end{document}
